% Options for packages loaded elsewhere
\PassOptionsToPackage{unicode}{hyperref}
\PassOptionsToPackage{hyphens}{url}
\PassOptionsToPackage{dvipsnames,svgnames,x11names}{xcolor}
%
\documentclass[
  letterpaper,
  DIV=11,
  numbers=noendperiod]{scrartcl}

\usepackage{amsmath,amssymb}
\usepackage{iftex}
\ifPDFTeX
  \usepackage[T1]{fontenc}
  \usepackage[utf8]{inputenc}
  \usepackage{textcomp} % provide euro and other symbols
\else % if luatex or xetex
  \usepackage{unicode-math}
  \defaultfontfeatures{Scale=MatchLowercase}
  \defaultfontfeatures[\rmfamily]{Ligatures=TeX,Scale=1}
\fi
\usepackage{lmodern}
\ifPDFTeX\else  
    % xetex/luatex font selection
    \setmainfont[]{Georgia}
\fi
% Use upquote if available, for straight quotes in verbatim environments
\IfFileExists{upquote.sty}{\usepackage{upquote}}{}
\IfFileExists{microtype.sty}{% use microtype if available
  \usepackage[]{microtype}
  \UseMicrotypeSet[protrusion]{basicmath} % disable protrusion for tt fonts
}{}
\makeatletter
\@ifundefined{KOMAClassName}{% if non-KOMA class
  \IfFileExists{parskip.sty}{%
    \usepackage{parskip}
  }{% else
    \setlength{\parindent}{0pt}
    \setlength{\parskip}{6pt plus 2pt minus 1pt}}
}{% if KOMA class
  \KOMAoptions{parskip=half}}
\makeatother
\usepackage{xcolor}
\ifLuaTeX
  \usepackage{luacolor}
  \usepackage[soul]{lua-ul}
\else
  \usepackage{soul}
  
\fi
\setlength{\emergencystretch}{3em} % prevent overfull lines
\setcounter{secnumdepth}{-\maxdimen} % remove section numbering
% Make \paragraph and \subparagraph free-standing
\makeatletter
\ifx\paragraph\undefined\else
  \let\oldparagraph\paragraph
  \renewcommand{\paragraph}{
    \@ifstar
      \xxxParagraphStar
      \xxxParagraphNoStar
  }
  \newcommand{\xxxParagraphStar}[1]{\oldparagraph*{#1}\mbox{}}
  \newcommand{\xxxParagraphNoStar}[1]{\oldparagraph{#1}\mbox{}}
\fi
\ifx\subparagraph\undefined\else
  \let\oldsubparagraph\subparagraph
  \renewcommand{\subparagraph}{
    \@ifstar
      \xxxSubParagraphStar
      \xxxSubParagraphNoStar
  }
  \newcommand{\xxxSubParagraphStar}[1]{\oldsubparagraph*{#1}\mbox{}}
  \newcommand{\xxxSubParagraphNoStar}[1]{\oldsubparagraph{#1}\mbox{}}
\fi
\makeatother

\usepackage{color}
\usepackage{fancyvrb}
\newcommand{\VerbBar}{|}
\newcommand{\VERB}{\Verb[commandchars=\\\{\}]}
\DefineVerbatimEnvironment{Highlighting}{Verbatim}{commandchars=\\\{\}}
% Add ',fontsize=\small' for more characters per line
\usepackage{framed}
\definecolor{shadecolor}{RGB}{241,243,245}
\newenvironment{Shaded}{\begin{snugshade}}{\end{snugshade}}
\newcommand{\AlertTok}[1]{\textcolor[rgb]{0.68,0.00,0.00}{#1}}
\newcommand{\AnnotationTok}[1]{\textcolor[rgb]{0.37,0.37,0.37}{#1}}
\newcommand{\AttributeTok}[1]{\textcolor[rgb]{0.40,0.45,0.13}{#1}}
\newcommand{\BaseNTok}[1]{\textcolor[rgb]{0.68,0.00,0.00}{#1}}
\newcommand{\BuiltInTok}[1]{\textcolor[rgb]{0.00,0.23,0.31}{#1}}
\newcommand{\CharTok}[1]{\textcolor[rgb]{0.13,0.47,0.30}{#1}}
\newcommand{\CommentTok}[1]{\textcolor[rgb]{0.37,0.37,0.37}{#1}}
\newcommand{\CommentVarTok}[1]{\textcolor[rgb]{0.37,0.37,0.37}{\textit{#1}}}
\newcommand{\ConstantTok}[1]{\textcolor[rgb]{0.56,0.35,0.01}{#1}}
\newcommand{\ControlFlowTok}[1]{\textcolor[rgb]{0.00,0.23,0.31}{\textbf{#1}}}
\newcommand{\DataTypeTok}[1]{\textcolor[rgb]{0.68,0.00,0.00}{#1}}
\newcommand{\DecValTok}[1]{\textcolor[rgb]{0.68,0.00,0.00}{#1}}
\newcommand{\DocumentationTok}[1]{\textcolor[rgb]{0.37,0.37,0.37}{\textit{#1}}}
\newcommand{\ErrorTok}[1]{\textcolor[rgb]{0.68,0.00,0.00}{#1}}
\newcommand{\ExtensionTok}[1]{\textcolor[rgb]{0.00,0.23,0.31}{#1}}
\newcommand{\FloatTok}[1]{\textcolor[rgb]{0.68,0.00,0.00}{#1}}
\newcommand{\FunctionTok}[1]{\textcolor[rgb]{0.28,0.35,0.67}{#1}}
\newcommand{\ImportTok}[1]{\textcolor[rgb]{0.00,0.46,0.62}{#1}}
\newcommand{\InformationTok}[1]{\textcolor[rgb]{0.37,0.37,0.37}{#1}}
\newcommand{\KeywordTok}[1]{\textcolor[rgb]{0.00,0.23,0.31}{\textbf{#1}}}
\newcommand{\NormalTok}[1]{\textcolor[rgb]{0.00,0.23,0.31}{#1}}
\newcommand{\OperatorTok}[1]{\textcolor[rgb]{0.37,0.37,0.37}{#1}}
\newcommand{\OtherTok}[1]{\textcolor[rgb]{0.00,0.23,0.31}{#1}}
\newcommand{\PreprocessorTok}[1]{\textcolor[rgb]{0.68,0.00,0.00}{#1}}
\newcommand{\RegionMarkerTok}[1]{\textcolor[rgb]{0.00,0.23,0.31}{#1}}
\newcommand{\SpecialCharTok}[1]{\textcolor[rgb]{0.37,0.37,0.37}{#1}}
\newcommand{\SpecialStringTok}[1]{\textcolor[rgb]{0.13,0.47,0.30}{#1}}
\newcommand{\StringTok}[1]{\textcolor[rgb]{0.13,0.47,0.30}{#1}}
\newcommand{\VariableTok}[1]{\textcolor[rgb]{0.07,0.07,0.07}{#1}}
\newcommand{\VerbatimStringTok}[1]{\textcolor[rgb]{0.13,0.47,0.30}{#1}}
\newcommand{\WarningTok}[1]{\textcolor[rgb]{0.37,0.37,0.37}{\textit{#1}}}

\providecommand{\tightlist}{%
  \setlength{\itemsep}{0pt}\setlength{\parskip}{0pt}}\usepackage{longtable,booktabs,array}
\usepackage{calc} % for calculating minipage widths
% Correct order of tables after \paragraph or \subparagraph
\usepackage{etoolbox}
\makeatletter
\patchcmd\longtable{\par}{\if@noskipsec\mbox{}\fi\par}{}{}
\makeatother
% Allow footnotes in longtable head/foot
\IfFileExists{footnotehyper.sty}{\usepackage{footnotehyper}}{\usepackage{footnote}}
\makesavenoteenv{longtable}
\usepackage{graphicx}
\makeatletter
\newsavebox\pandoc@box
\newcommand*\pandocbounded[1]{% scales image to fit in text height/width
  \sbox\pandoc@box{#1}%
  \Gscale@div\@tempa{\textheight}{\dimexpr\ht\pandoc@box+\dp\pandoc@box\relax}%
  \Gscale@div\@tempb{\linewidth}{\wd\pandoc@box}%
  \ifdim\@tempb\p@<\@tempa\p@\let\@tempa\@tempb\fi% select the smaller of both
  \ifdim\@tempa\p@<\p@\scalebox{\@tempa}{\usebox\pandoc@box}%
  \else\usebox{\pandoc@box}%
  \fi%
}
% Set default figure placement to htbp
\def\fps@figure{htbp}
\makeatother
% definitions for citeproc citations
\NewDocumentCommand\citeproctext{}{}
\NewDocumentCommand\citeproc{mm}{%
  \begingroup\def\citeproctext{#2}\cite{#1}\endgroup}
\makeatletter
 % allow citations to break across lines
 \let\@cite@ofmt\@firstofone
 % avoid brackets around text for \cite:
 \def\@biblabel#1{}
 \def\@cite#1#2{{#1\if@tempswa , #2\fi}}
\makeatother
\newlength{\cslhangindent}
\setlength{\cslhangindent}{1.5em}
\newlength{\csllabelwidth}
\setlength{\csllabelwidth}{3em}
\newenvironment{CSLReferences}[2] % #1 hanging-indent, #2 entry-spacing
 {\begin{list}{}{%
  \setlength{\itemindent}{0pt}
  \setlength{\leftmargin}{0pt}
  \setlength{\parsep}{0pt}
  % turn on hanging indent if param 1 is 1
  \ifodd #1
   \setlength{\leftmargin}{\cslhangindent}
   \setlength{\itemindent}{-1\cslhangindent}
  \fi
  % set entry spacing
  \setlength{\itemsep}{#2\baselineskip}}}
 {\end{list}}
\usepackage{calc}
\newcommand{\CSLBlock}[1]{\hfill\break\parbox[t]{\linewidth}{\strut\ignorespaces#1\strut}}
\newcommand{\CSLLeftMargin}[1]{\parbox[t]{\csllabelwidth}{\strut#1\strut}}
\newcommand{\CSLRightInline}[1]{\parbox[t]{\linewidth - \csllabelwidth}{\strut#1\strut}}
\newcommand{\CSLIndent}[1]{\hspace{\cslhangindent}#1}

\KOMAoption{captions}{tableheading}
\makeatletter
\@ifpackageloaded{caption}{}{\usepackage{caption}}
\AtBeginDocument{%
\ifdefined\contentsname
  \renewcommand*\contentsname{Table of contents}
\else
  \newcommand\contentsname{Table of contents}
\fi
\ifdefined\listfigurename
  \renewcommand*\listfigurename{List of Figures}
\else
  \newcommand\listfigurename{List of Figures}
\fi
\ifdefined\listtablename
  \renewcommand*\listtablename{List of Tables}
\else
  \newcommand\listtablename{List of Tables}
\fi
\ifdefined\figurename
  \renewcommand*\figurename{Figure}
\else
  \newcommand\figurename{Figure}
\fi
\ifdefined\tablename
  \renewcommand*\tablename{Table}
\else
  \newcommand\tablename{Table}
\fi
}
\@ifpackageloaded{float}{}{\usepackage{float}}
\floatstyle{ruled}
\@ifundefined{c@chapter}{\newfloat{codelisting}{h}{lop}}{\newfloat{codelisting}{h}{lop}[chapter]}
\floatname{codelisting}{Listing}
\newcommand*\listoflistings{\listof{codelisting}{List of Listings}}
\makeatother
\makeatletter
\makeatother
\makeatletter
\@ifpackageloaded{caption}{}{\usepackage{caption}}
\@ifpackageloaded{subcaption}{}{\usepackage{subcaption}}
\makeatother

\usepackage{bookmark}

\IfFileExists{xurl.sty}{\usepackage{xurl}}{} % add URL line breaks if available
\urlstyle{same} % disable monospaced font for URLs
\hypersetup{
  pdftitle={Quarto Document Template},
  pdfauthor={Kaylynn Hiller},
  colorlinks=true,
  linkcolor={blue},
  filecolor={Maroon},
  citecolor={Blue},
  urlcolor={Blue},
  pdfcreator={LaTeX via pandoc}}


\title{Quarto Document Template}
\author{Kaylynn Hiller}
\date{2026-01-16}

\begin{document}
\maketitle


\section{\texorpdfstring{1\textsuperscript{st}
Header}{1st Header}}\label{st-header}

Vestibulum sollicitudin quis leo id ornare. Aenean tristique auctor
justo sed ornare. Morbi ac suscipit eros. Integer tortor enim, cursus
vitae tincidunt vitae, tristique id lectus. Donec mollis mauris ac
sapien venenatis cursus. Suspendisse dapibus nunc augue. Nam in urna
egestas elit vehicula malesuada nec non tortor. Morbi ac tincidunt
velit, sed viverra metus. Phasellus lacinia pellentesque tellus, eget
finibus nisl commodo vitae. Curabitur ac nisi vitae nunc eleifend
mollis. In id dapibus sem, vel lobortis odio. Suspendisse tincidunt
fringilla scelerisque. Interdum et malesuada fames ac ante ipsum primis
in faucibus. Sed et leo lectus. Cras felis arcu, ornare nec arcu cursus,
commodo venenatis velit.

\subsection{\texorpdfstring{2\textsuperscript{nd}
Header}{2nd Header}}\label{nd-header}

Vestibulum scelerisque sed erat eu porta. Nullam sit amet dictum quam.
Ut risus quam, aliquam non pellentesque at, aliquet eget nisi. Sed
molestie sapien nec interdum volutpat. Sed suscipit leo ut efficitur
vehicula. Suspendisse ac sodales risus, eu pulvinar justo. Donec magna
nibh, ultrices id tortor ac, accumsan finibus mauris. Class aptent
taciti sociosqu ad litora torquent per conubia nostra, per inceptos
himenaeos. Vestibulum finibus elit augue, non porta turpis eleifend ac.
Aliquam aliquet at nulla in sollicitudin. Suspendisse laoreet dolor id
placerat interdum. Phasellus a metus blandit, lacinia arcu aliquet,
ultrices urna. Suspendisse blandit odio eget nulla cursus, ac fermentum
libero tincidunt.

\subsubsection{\texorpdfstring{3\textsuperscript{rd}
Header}{3rd Header}}\label{rd-header}

Nullam eu diam vel ipsum malesuada gravida. Vestibulum id vestibulum
dui. Donec tincidunt euismod sapien dignissim lobortis. Cras et velit eu
orci porttitor blandit. Nulla nec nunc ac ex mollis luctus. Duis nisl
felis, finibus vel justo vel, sodales condimentum nibh. Pellentesque
viverra egestas elit, at venenatis purus finibus et. Suspendisse auctor
sem in nisl tincidunt vehicula. Pellentesque habitant morbi tristique
senectus et netus et malesuada fames ac turpis egestas.

\subsection{Text formatting}\label{text-formatting}

\begin{quote}
I want to be a part of the people that make meaning, not the thing that
is made.

- Barbie, The Movie
\end{quote}

\emph{italic} \textbf{bold} \st{strikeout} \texttt{code}

superscript\textsuperscript{2} subscript\textsubscript{2}

\ul{underline} \textsc{small caps}

\subsection{Lists}\label{lists}

\begin{itemize}
\item
  Bulleted list item 1
\item
  Item 2

  \begin{itemize}
  \item
    Item 2a
  \item
    Item 2b
  \end{itemize}
\end{itemize}

\begin{enumerate}
\def\labelenumi{\arabic{enumi}.}
\item
  Numbered list item 1
\item
  Item 2. The numbers are incremented automatically in the output.
\item
  Another numbered list item
\end{enumerate}

\subsection{Images and Figures}\label{images-and-figures}

Quarto allows for
\href{https://quarto.org/docs/authoring/cross-references.html\#figures}{easy
cross-references of images and plots}. Figure~\ref{fig-dog} shows an
image, whereas Figure~\ref{fig-plot} shows a plot created using ggplot2.

\begin{figure}

\centering{

\pandocbounded{\includegraphics[keepaspectratio]{images/IMG_7135.jpeg}}

}

\caption{\label{fig-dog}This is a picture of my dogs enjoying the sun
outside.}

\end{figure}%

\begin{figure}

\centering{

\pandocbounded{\includegraphics[keepaspectratio]{quarto_template_files/figure-pdf/fig-plot-1.pdf}}

}

\caption{\label{fig-plot}This plot shows the effect of vitamin C on
tooth growth in Guinea Pigs. The response is the length of odontoblasts
(cells responsible for tooth growth) in 60 guinea pigs. Each animal
received one of three dose levels of vitamin C (0.5, 1, and 2 mg/day) by
one of two delivery methods, orange juice or ascorbic acid (a form of
vitamin C and coded as \texttt{VC}).}

\end{figure}%

To find more about the dataset used to create Figure~\ref{fig-plot} type
the following line of code into the console:

\begin{Shaded}
\begin{Highlighting}[]
\NormalTok{?ToothGrowth}
\end{Highlighting}
\end{Shaded}

\subsection{Equations}\label{equations}

It's also easy to cross-reference equations. For example,
Equation~\ref{eq-1} depicts a \emph{normal distribution.}

\begin{equation}\phantomsection\label{eq-1}{
f(x) = \frac{1}{\sqrt{2\pi\sigma^2}} e^{-\frac{(x-\mu)^2}{2\sigma^2}}
}\end{equation}

\subsection{Tables}\label{tables}

We can also cross-reference
\href{https://quarto.org/docs/authoring/cross-references.html\#tables}{tables}
easily, as shown in Table~\ref{tbl-manual} and
Table~\ref{tbl-regression}.

\begin{longtable}[]{@{}ll@{}}
\caption{This table was created using
Markdown.}\label{tbl-manual}\tabularnewline
\toprule\noalign{}
Col1 & Col2 \\
\midrule\noalign{}
\endfirsthead
\toprule\noalign{}
Col1 & Col2 \\
\midrule\noalign{}
\endhead
\bottomrule\noalign{}
\endlastfoot
a & b \\
c & d \\
\end{longtable}

Table~\ref{tbl-regression} shows some very simple regression models.

\newpage

\begin{verbatim}
[1] "data.frame"
\end{verbatim}

\begin{table}

\centering{

\centering
\begin{talltblr}[         %% tabularray outer open
entry=none,label=none,
note{}={+ p \num{< 0.1}, * p \num{< 0.05}, ** p \num{< 0.01}, *** p \num{< 0.001}},
]                     %% tabularray outer close
{                     %% tabularray inner open
colspec={Q[]Q[]Q[]Q[]Q[]},
hline{2}={1-5}{solid, black, 0.05em},
hline{12}={1-5}{solid, black, 0.05em},
hline{1}={1-5}{solid, black, 0.1em},
hline{14}={1-5}{solid, black, 0.1em},
column{2-5}={}{halign=c},
column{1}={}{halign=l},
}                     %% tabularray inner close
& Model 1 & Model 2 & Model 3 & Model 4 \\
(Intercept) & \num{17.147}*** & \num{14.594}*** & \num{23.334}*** & \num{24.832}*** \\
& (\num{1.125}) & (\num{0.926}) & (\num{2.233}) & (\num{2.890}) \\
am & \num{7.245}*** & \num{6.067}*** & \num{5.299}*** & \num{4.419}** \\
& (\num{1.764}) & (\num{1.275}) & (\num{1.038}) & (\num{1.493}) \\
vs &  & \num{6.929}*** & \num{2.659}+ & \num{2.052} \\
&  & (\num{1.262}) & (\num{1.442}) & (\num{1.627}) \\
hp &  &  & \num{-0.045}*** & \num{-0.038}* \\
&  &  & (\num{0.011}) & (\num{0.014}) \\
disp &  &  &  & \num{-0.008} \\
&  &  &  & (\num{0.010}) \\
Num.Obs. & \num{32} & \num{32} & \num{32} & \num{32} \\
R2 & \num{0.360} & \num{0.686} & \num{0.806} & \num{0.810} \\
\end{talltblr}

}

\caption{\label{tbl-regression}This table was created using the
\texttt{modelsummary} package}

\end{table}%

\emph{Ask Steve what he thinks about those asterisks!}

\newpage

\subsection{Citations}\label{citations}

Finally, Quarto has great Zotero integration.

\begin{quote}
Hi! My name is Andrés. One of my favorite academic articles is
({``Advocating for Women's Rights to Evidence-Based Care and Childbirth
Alternatives in the United States: The Time Is Now - ProQuest''}).
(Declercq et al.) is a recent article I read. The next thing I plan on
reading is this article (Elwyn et al.).
\end{quote}

\subsection{References}\label{references}

\phantomsection\label{refs}
\begin{CSLReferences}{1}{0}
\bibitem[\citeproctext]{ref-advocati}
\href{https://search.proquest.com/docview/212811884?pq-origsite=gscholar&fromopenview=true}{Advocating
for women's rights to evidence-based care and childbirth alternatives in
the united states: The time is now - ProQuest}.

\bibitem[\citeproctext]{ref-declercq}
Declercq, Eugene R, Carol Sakala, Maureen P Corry, Sandra Applebaum, and
Ariel Herrlich.{``Report of the Third National U.S. Survey of Women{'}s
Childbearing Experiences.''} 94.

\bibitem[\citeproctext]{ref-elwyn}
Elwyn, Glyn, Adrian Edwards, Richard Gwyn, and Richard Grol.{``Towards a
Feasible Model for Shared Decision Making: Focus Group Study with
General Practice Registrars.''} \emph{BMJ} 318 (7212): 753--56. DOI:
\url{https://doi.org/10.1136/bmj.319.7212.753}.

\end{CSLReferences}




\end{document}
